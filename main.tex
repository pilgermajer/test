\documentclass[oneside,a4paper,12pt]{article}
\renewcommand{\baselinestretch}{1.5} 
\usepackage{indentfirst}
\setlength{\parindent}{5mm}
\usepackage{mathptmx}
\usepackage{amsmath}
\usepackage[utf8]{inputenc}
\usepackage{color}
\usepackage[top=40mm, bottom=25mm, left=35mm, right=25mm, paperwidth=210mm, paperheight=297mm]{geometry}
\usepackage{graphicx}
\usepackage{ltxtable, tabularx, longtable} 
\usepackage{booktabs}
\usepackage{amsmath}
\usepackage{amssymb}
\usepackage[magyar]{babel}
\usepackage{listings}
\usepackage{setspace}
\usepackage{fancyhdr}
\pagestyle{fancy}
\rhead{Szakdolgozat címe}
\lhead{}
%\usepackage{showframe}
\title {Diplomamunka}
%\author{Szerző:Bencsik Gergely\\Témavezető: Dr. Sáti Zoltán PhD.}
%\subtitle{syncEQ - Elektroakusztikai adaptív szűrő}
\setlength{\hoffset}{-1in} 
\setlength{\voffset}{-1in}
\topmargin=00mm
\oddsidemargin=35mm
\headheight=30mm
\headsep=10mm
\marginparsep=0mm
\marginparwidth=0mm
\textwidth=150mm
\textheight=232mm
\begin{document}
	%\setcounter{tocdepth}{0}
	%\pagenumbering{arabic}
	\thispagestyle{empty}
	\begin{center}
		\LARGE
		TÖRZSSZÁM\\
		\Huge
		\vspace{80mm}
		DIPLOMAMUNKA\\
		\vspace{90mm}
		\Large
		Gipsz Jakab\\
		\bigskip
		2019
	\end{center}
	\pagebreak
	\thispagestyle{empty}
	\begin{center}
		\Large
		\begin{spacing}{1}
			Pécsi Tudományegyetem\\
			Műszaki és Informatikai Kar\\
			Mérnök Informatikus MSc. Szak\\
		\end{spacing}
		\Huge
		\vspace{80mm}
		{\bf DIPLOMAMUNKA}\\
		\bigskip
		\Large
		Szakdolgozat Címe\\
		\vspace{60mm}
		\Large
		\begin{spacing}{1}
			Készítette: Gipsz Jakab\\
			Témavezető: Dr. Szürke Gandalf\\
		\end{spacing}
		\bigskip
		\bigskip
		Pécs
	\end{center}
	
	\pagebreak
	\thispagestyle{empty}
	
	\begin{center}
		\bf
		\begin{tabularx}{\textwidth}{
				>{\hsize=0.5\hsize}X
				>{\hsize=0.2\hsize}X
				>{\hsize=0.3\hsize}X}
			PÉCSI TUDOMÁNYEGYETEM & & Diplomamunka száma:\\
			MŰSZAKI ÉS INFORMATIKAI KAR & & TÖRZSSZÁM\\
			Mérnök Informatikus MSc. Szak & & \\
		\end{tabularx}
	\end{center}
	\vspace{20mm}
	\begin{center}
		\bf
		Diplomamunka\\
		\bigskip\bigskip
		..................................................................................\\
		hallgató részére
		\bigskip\bigskip
	\end{center}
	\noindent
	A záróvizsgát megelőzően diplomamunkát kell benyújtania, amelynek témáját és feladatait az alábbiak szerint határozom meg:\\
	
	\noindent
	\textbf{Téma:}\\
	Téma címe.\\
	
	\noindent
	\textbf{Feladat:}\\
	Feladat leírása.\\
	
	\noindent
	A diplomamunka készítéséért felelős tanszék:\\
	............................................................................................\\
	
	\noindent
	Külső konzulens:\\ ............................................................................................
	\\munkahelye:\\ ............................................................................................\\
	
	\noindent
	Témavezető: Dr. Szürke Gandalf\\
	munkahelye: Pécsi Tudományegyetem, Műszaki és Informatikai Kar
	\\
	\bigskip
	\bigskip
	\begin{center}
		\begin{tabularx}{\textwidth}{
				>{\hsize=0.5\hsize}X
				>{\hsize=0.3\hsize}X
				>{\hsize=0.2\hsize}X}
			Pécs, 2019. .................. & & .............................. \\
			& & Dr. Iványi Péter
		\end{tabularx}
	\end{center}
	
	\pagebreak
	\thispagestyle{empty}
	
	\begin{center}
		\Large
		HALLGATÓI NYILATKOZAT\\
	\end{center}
	\vspace{30mm}
	\noindent \normalsize
	Alulírott szigorló hallgató kijelentem, hogy a diplomamunka saját munkám eredménye. A felhasznált szakirodalmat és eszközöket azonosíthatóan közöltem. Egyéb jelentősebb segítséget nem vettem igénybe.\\
	Az elkészült diplomamunkában talált eredményeket a feladatot kiíró intézmény saját céljaira térítés nélkül felhasználhatja.\\
	\vspace{30mm}
	\begin{center}
		\begin{tabularx}{\textwidth}{
				>{\hsize=0.5\hsize}X
				>{\hsize=0.3\hsize}X
				>{\hsize=0.2\hsize}X}
			Pécs, 2019. .................. & & .............................. \\
			& & Gipsz Jakab
		\end{tabularx}
	\end{center}
	
	\pagebreak
	\pagenumbering{Roman}
	\setcounter{page}{1}
	\tableofcontents
	
	\pagebreak
	
	\pagenumbering{arabic}
	\setcounter{page}{1}
	
	\section{Köszönetnyilvánítás}
	Köszönetnyílvánítás helye...
	
	\pagebreak
	
	\section{Előszó}
	Előszó helye...
	\pagebreak
	
	\section{Példák}
	
	\subsection{Ábra}
	
	\begin{center}
		\includegraphics[scale=3]{ptemik.jpg}\\
		1. ábra. Ábra beillesztésének módja.\\
	\end{center}
	
	\subsection{Lista}
	
	\subsubsection{Rendezetlen lista}
	\begin{itemize}
		\item Első pont.
		\item Második pont.
		\item Harmadik pont.
	\end{itemize}
	
	\subsubsection{Számozott lista}
	\begin{enumerate}
		\item Első pont.
		\item Második pont.
		\item Harmadik pont.
	\end{enumerate}
	
	\subsubsection{Kombinált lista}
	\begin{enumerate}
		\item Első pont.
		\item Második pont.
		\item Harmadik pont.
		\begin{enumerate}
			\item Első pont.
			\item Második pont.
			\item Harmadik pont.
		\end{enumerate}
	\end{enumerate}
	
	\subsection{Táblázatok}
	
	Érdemes az alábbi példákbaban áttekinteni a táblázat paraméterezését. Egy-egy karakter vagy sor hozzáadásával könnyen átalakítható és pontosan méretezhető a táblázat.
	\footnote{Kérdésekkel bátran tessék keresni: becsegergely@gmail.com - Bencsik Gergely. (Ha a google nem segített.)}
	
	\bigskip
	
	\begin{center}
		\small
		\begin{tabularx}{\textwidth}{
				|>{\hsize=0.5\hsize}X|
				>{\hsize=0.5\hsize}X|}
			\hline
			1 & 2 \\ \hline
			3 & 4 \\ \hline
		\end{tabularx}
	\end{center}
	
	\begin{center}
		\small
		\begin{tabularx}{\textwidth}{
				>{\hsize=0.5\hsize}X|
				>{\hsize=0.5\hsize}X}
			1 & 2 \\ \hline
			3 & 4 \\ 
		\end{tabularx}
	\end{center}
	
	\begin{center}
		\small
		\begin{tabularx}{\textwidth}{
				>{\hsize=0.1\hsize}X|
				>{\hsize=0.45\hsize}X
				>{\hsize=0.45\hsize}X|}
			1 & 2 & 3 \\ \hline
			4 & 5 & 6 \\ 
		\end{tabularx}
	\end{center}
	
	\subsection{Forráskód}
	
	\begin{lstlisting}[
		language=C,
		frame=single,
		showspaces=false,
		showstringspaces=false,
		numbers=left,
		numbersep=2mm,
		breaklines=true,
		tabsize=1,
		title=Forrás: main.c]
		#include <...>
		
		void main()
		{
			...
		}
	\end{lstlisting}
	
	\subsection{Formula}
	
	$$ E_t = \dfrac{1}{2\pi} \int_{-\infty}^{\infty} |\underline{H}(\omega)^2|\, \mathrm{d}\omega $$
	
	\bigskip
	
	\pagebreak
	\section{Irodalomjegyzék}
	
	\begin{enumerate}
		\item Cím1
		\item Cím2
	\end{enumerate}
	
\end{document}

